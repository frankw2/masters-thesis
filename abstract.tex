% $Log: abstract.tex,v $
% Revision 1.1  93/05/14  14:56:25  starflt
% Initial revision
% 
% Revision 1.1  90/05/04  10:41:01  lwvanels
% Initial revision
% 
%
%% The text of your abstract and nothing else (other than comments) goes here.
%% It will be single-spaced and the rest of the text that is supposed to go on
%% the abstract page will be generated by the abstractpage environment.  This
%% file should be \input (not \include 'd) from cover.tex.
In today's web, applications whose
value depends on access to user
data, such as quantified self applications
and applications that analyze medical
records, suffer when user data is 
locked in application servers.
In this thesis, we present Sieve,
a new system that solves
this problem by centralizing
user data. It provides
secure, delegated access to a user's
sensitive cloud data. Sieve enforces
\emph{cryptographically strong
restrictions} on how third party
web services can access that data.
However, Sieve can still be compatible
with monetization systems like targeted
advertising, reducing the barrier to
adoption. In Sieve, each user uploads
her data in encrypted form to a
cloud-based storage provider. 

Each data object is associated with
attributes like file type, subject
matter, and associated user names;
these attributes arise from automatic
annotation or manual user tagging.
When a web service requests access
to the user's data, she generates
a service-specific access policy.
This policy is expressed in terms of
attributes and simple operators like
equals and less-than. Using attribute-based
encryption, Sieve automatically translates
the human-readable access policy into a
public/private key pair that is given
to the web service. The key pair allows
the web service to independently access
and decrypt the delegated user objects
(but no others). Unlike other systems
that use attribute based encryption,
Sieve provides strong revocation semantics
and distributed trust. Using this scheme,
Sieve provides users with true control
over how their cloud data is accessed.
This contrasts with popular delegation
schemes like OAuth in which policies
are written by web services and lacking
in cryptographically strong protections.
