% $Log: abstract.tex,v $
% Revision 1.1  93/05/14  14:56:25  starflt
% Initial revision
% 
% Revision 1.1  90/05/04  10:41:01  lwvanels
% Initial revision
% 
%
%% The text of your abstract and nothing else (other than comments) goes here.
%% It will be single-spaced and the rest of the text that is supposed to go on
%% the abstract page will be generated by the abstractpage environment.  This
%% file should be \input (not \include 'd) from cover.tex.
Modern web services rob users of low-level
control over cloud storage; a user's single
logical data set is scattered across multiple
storage silos whose access controls are set
by the web services, not users. The result
is that users lack the ultimate authority
to determine how their data is shared with
other web services.

In this thesis, we introduce Sieve, a new
architecture for selectively exposing user
data to third party web services in a provably
secure manner. Sieve
starts with a user-centric storage model:
each user uploads encrypted data to a
single cloud store, and by default, only
the user knows the decryption keys. Given
this storage model, Sieve defines an
infrastructure to support rich, legacy web
applications. Using attribute-based encryption,
Sieve allows users to define intuitive, 
understandable access policies that are
cryptographically enforceable. Using key
homomorphism, Sieve can re-encrypt user
data on storage providers in situ, revoking
decryption keys from web services without
revealing new ones to the storage provider.
Using secret sharing and two-factor authentication,
Sieve protects against the loss of user
devices like smartphones and laptops. The
result is that users can enjoy rich, legacy
web applications, while benefitting from
cryptographically strong controls over
what data the services can access.