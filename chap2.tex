\chapter{Threat Model}
\label{sec:threatModel}

We focus on three kinds of principals. A
\textbf{user} is someone who wants to store
data online and selectively expose it
to a \textbf{third party web service}. The
user stores her (encrypted) data on a
\textbf{cloud storage provider}. Each
user has one storage provider, but
potentially many third parties which
need delegated access. Potential storage
providers include Amazon EC2 and Microsoft
Azure. Potential third party web services
are FitBit, Lark, Misfit, Mint, and 
any other application that generates new value
from sensitive user data.

The user has a financial agreement with
the storage provider: the user pays for the provider 
to keep her objects and to participate in the
Sieve protocol on her behalf.
The user places encrypted data on the
storage provider, but never reveals
the decryption keys to the provider. 
This protects the confidentiality of
user data if the storage service is
malicious or compromised. Using signatures,
Sieve can also protect the data's
integrity. However, Sieve cannot
guarantee the availability or freshness
of the data that the storage provider
delivers to a web service. If desired,
Sieve can be layered atop storage
systems like CloudProof~\cite{cloudproof}
which do provide those properties.

Sieve does not hide access patterns
or object metadata (i.e., attributes) from
the storage provider. Thus, a curious
provider can learn which encrypted
objects a third party has been authorized
to read, as well as the attributes
that are associated with those objects.
If users are concerned about data
leakage via access patterns, they
could layer Sieve atop an ORAM
protocol~\cite{shroud}. To hide attributes,
Sieve could also use predicate encryption~\cite{katz2008,shen2009}.
However, ORAM and predicate encryption
incur heavy performance overheads
(\S\ref{sec:related}), so Sieve uses
lighter weight mechanisms that reduce
service latency at the cost of leaking
more metadata. We believe that this
trade-off is reasonable for many users
and companies, given the importance
of low latencies in modern web services~\cite{wprof,bobtail}.

With respect to third party web services,
Sieve's goal is to only reveal user data
as permitted by the user's disclosure
policies. After Sieve transmits information
to a third party web server, Sieve cannot
restrict what the third party does with that
data, since third parties may cache user data
locally, even after the user has revoked access
to the canonical versions that reside on her
storage server. Third parties can also share 
decrypted user data with other
principals via out-of-band, non-Sieve protocols
like in the current web infrastructure.
Solving these issues is beyond the scope of
this paper. However, if a web service shares its user-issued
ABE key, Sieve can revoke it, preventing
those parties from downloading old or new
user data from the storage provider
(\S\ref{sec:revocation}).

Sieve does not protect against client-side
attacks like social engineering~\cite{socialEngineering}
or cross-site scripting~\cite{xss}.
However, Sieve uses secret sharing to
protect system-wide secrets like the ABE
master key from the loss of a single device
(\S\ref{sec:secretSharing}).
